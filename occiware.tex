\documentclass[a4paper,11pt,twoside]{report}
\usepackage{lipsum}
\usepackage{hyperref}
\usepackage[T1]{fontenc}
%\usepackage[frenchb]{babel}
%\usepackage[french,english]{babel}
\usepackage{occiware}

\owdelivnr{D0.0.1}
\owexpected{2015/05/31}
\owdelivered{\today}
%\owstatus{Draft}
%\ownature{Public}
\owversion{0.1}

\begin{document}
\selectlanguage{french}

\title{OCCIware delivery template}

\author{Jean Parpaillon}

%\thanks{Thank you to ...}

\date{\today}

\maketitle

\begin{abstract}
  Le document est un exemple d'utilisation du package \LaTeX
  ``occiware'' et des macros fournies. Ce package doit permettre de
  rédiger les rapports (``livrables'') du projet \occiware.
\end{abstract}

\chapter{Le package \LaTeX ``occiware''}
\label{chap:one}

\section{Utilisation}
\label{sec:utilisation}

Le fichier \texttt{occiware.sty} doit être présent dans le chemin de
compilation du document.

Inclure le package dans le document avec la commande:

\begin{verbatim}
\usepackage{occiware}
\end{verbatim}

Les langues disponibles pour le template sont:
\begin{itemize}
\item français,
\item anglais (défaut)
\end{itemize}

Pour sélectionner la langue du document, inclure \emph{après}
  \verb+\begin{document}+ la commande:

\begin{verbatim}
\selectlanguage{french}
\end{verbatim}

\section{Macros}
\label{sec:macros}

\subsection{Macros à définir}

\begin{itemize}
\item \verb+\owdelivnr+ : numéro du livrable (ex.: 2.1.1)
\item \verb+\owexpected+ : date attendue du livrable
\item \verb+\owdelivered+ : date de livraison (défaut: \verb+\today+)
\item \verb+\owstatus+ : \textit{draft} | \textit{final} (défaut: \textit{draft})
\item \verb+\ownature+ : \textit{public} | \textit{private} (défaut: \textit{public})
\item \verb+\owversion+ : version du document
\end{itemize}

\subsection{Trigrammes des partenaires}

\begin{itemize}
\item \verb+\OPW+ : \OPW
\item \verb+\AON+ : \AON
\item \verb+\CSR+ : \CSR
\item \verb+\TSP+ : \TSP
\item \verb+\INR+ : \INR
\item \verb+\LNG+ : \LNG
\item \verb+\OBO+ : \OBO
\item \verb+\OWC+ : \OWC
\item \verb+\PNQ+ : \PNQ
\item \verb+\UJF+ : \UJF
\end{itemize}

\subsection{Citations}

\begin{itemize}
\item \cite{occi-core-12}
\item \cite{occi-http-rendering}
\end{itemize}

\chapter{Chapter Two}
\label{chap:two}

\section{An Example Section}
\label{sec:example}

\lipsum[1-5]

\section{Another Very Interesting Section}
\label{sec:interesting}

\subsection{Lorem 1}
\label{sec:lorem1}

\lipsum[10-15]

\subsection{Lorem 2}
\label{sec:lorem2}

\lipsum[16-21]


\bibliographystyle{ieeetr}
\bibliography{occiware}

\end{document}
